\documentclass{article}

\usepackage[a4paper]{geometry}

\usepackage[T1,T2A]{fontenc}
\usepackage[utf8]{inputenc}
\usepackage[russian]{babel}

\usepackage[useregional]{datetime2}

\usepackage{inconsolata}

\usepackage[colorlinks]{hyperref}

\newcommand{\latinttfamily}{\fontencoding{T1}\ttfamily}
\DeclareTextFontCommand{\ltexttt}{\latinttfamily}
\makeatletter
\renewcommand*{\verbatim@font}{\latinttfamily}
\makeatother

\title{Предложение 2: line-based tokens \\ (ревизия 0)}
\author{Авторы: комитет Б05-831 МФТИ \\ Редактор: andtit2001}
\date{\DTMdate{2019-11-07}}

\begin{document}
\maketitle
\tableofcontents
\pagebreak

\section{Назначение}
Настоящее предложение определяет лексические соглашения (см.~документ D0, секция 3) для использования в формате MIPT.

\section{Предлагаемые определения}
\subsection{Строка}
\begin{enumerate}
	\item Название токена: \ltexttt{string}
	\item Описание: произвольная последовательность символов в кодировке UTF-8 длины не меньше 2, в которой есть ровно один символ новой строки (newline character), расположенный в конце.
	\item Регулярное выражение: \verb/[^\n]+\n/
\end{enumerate}

\subsection{Открывающий разделитель}
\begin{enumerate}
	\item Название токена: \ltexttt{open-delim}
	\item Описание: левая фигурная скобка (left curly bracket), после которой идёт символ новой строки.
	\item Регулярное выражение: \verb/{\n/
\end{enumerate}

\subsection{Закрывающий разделитель}
\begin{enumerate}
	\item Название токена: \ltexttt{close-delim}
	\item Описание: правая фигурная скобка (right curly bracket), после которой идёт символ новой строки.
	\item Регулярное выражение: \verb/}\n/
\end{enumerate}
\end{document}
