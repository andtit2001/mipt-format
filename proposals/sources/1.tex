\documentclass{article}

\usepackage[a4paper]{geometry}

\usepackage[T1,T2A]{fontenc}
\usepackage[utf8]{inputenc}
\usepackage[russian]{babel}

\usepackage[useregional]{datetime2}

\usepackage{inconsolata}

\usepackage[colorlinks]{hyperref}

\newcommand{\latinttfamily}{\fontencoding{T1}\ttfamily}
\DeclareTextFontCommand{\ltexttt}{\latinttfamily}
\makeatletter
\renewcommand*{\verbatim@font}{\latinttfamily}
\makeatother

\title{Предложение 1: quote-based strings (и другие токены) \\ (ревизия 0)}
\author{Авторы: комитет Б05-831 МФТИ \\ Редактор: andtit2001}
\date{\DTMdate{2019-11-07}}

\begin{document}
\maketitle
\tableofcontents
\pagebreak

\section{Назначение}
Настоящее предложение определяет лексические соглашения (см.~документ D0, секция 3) для использования в формате MIPT.

\section{Предлагаемые определения}
\subsection{Строка}
\begin{enumerate}
	\item Название токена: \ltexttt{string}
	\item Описание: произвольная последовательность символов в кодировке UTF-8 длины не меньше 3, которая начинается и заканчивается двойной кавычкой (quotation mark), причём все прочие вхождения двойной кавычки предварены обратной косой чертой (backslash). Остальные вхождения обратной косой черты всегда можно разбить на непересекающиеся подстроки длины 2.
	\item Регулярное выражение: \verb/"(?:[^\\"]|\\[\\"])+"/
\end{enumerate}

\subsection{Открывающий разделитель}
\begin{enumerate}
	\item Название токена: \ltexttt{open-delim}
	\item Описание: левая фигурная скобка (left curly bracket).
	\item Регулярное выражение: \verb/{/
\end{enumerate}

\subsection{Закрывающий разделитель}
\begin{enumerate}
	\item Название токена: \ltexttt{close-delim}
	\item Описание: правая фигурная скобка (right curly bracket).
	\item Регулярное выражение: \verb/}/
\end{enumerate}
\end{document}
